\section{Blockchain interface}
\label{sec:blockchain-interface}

\newcommand{\DIEnv}{\type{DIEnv}}
\newcommand{\DIState}{\type{DIState}}

\newcommand{\UPIEnv}{\type{UPIEnv}}
\newcommand{\UPIState}{\type{UPIState}}

\subsection{Delegation interface}
\label{sec:delegation-interface}

\begin{figure}[htb]
  \emph{Delegation interface environments}
  \begin{equation*}
    \DIEnv =
    \left(
      \begin{array}{r@{~\in~}lr}
        \mathcal{K} & \powerset{\VKeyGen} & \text{allowed delegators}\\
        \var{e} & \Epoch & \text{epoch}\\
        \var{s} & \Slot & \text{slot}\\
        \var{d} & \SlotCount & \text{certificate liveness parameter}
      \end{array}
    \right)
  \end{equation*}

  \emph{Delegation interface states}
  \begin{equation*}
    \DIState
    = \left(
      \begin{array}{r@{~\in~}lr}
        \var{dms} & \VKeyGen \mapsto \VKey & \text{delegation map}\\
        \var{dws} & \VKeyGen \mapsto \Slot & \text{when last delegation occurred}\\
        \var{sds} & \seqof{(\Slot \times (\VKeyGen \times \VKey))} & \text{scheduled delegations}\\
        \var{eks} & \powerset{(\Epoch \times \VKeyGen)} & \text{key-epoch delegations}
      \end{array}
    \right)
  \end{equation*}

  \emph{Delegation transitions}
  \begin{equation*}
    \_ \vdash \_ \trans{deleg}{\_} \_ \in
    \powerset (\DIEnv \times \DIState \times \seqof{\DCert} \times \DIState)
  \end{equation*}
  \caption{Delegation interface transition-system types}
  \label{fig:ts-types:delegation-interface}
\end{figure}

\subsubsection{Delegation interface rules}
\label{sec:delegation-interface-rules}

\begin{figure}[htb]
  \begin{equation}
    \label{eq:rule:delegation-interface}
    \inference
    {
      {\begin{array}{l}
         \mathcal{K} \\
         e\\
         s\\
         d
       \end{array}}
      \vdash
      {
        \left(
          \begin{array}{l}
            \var{sds}\\
            \var{eks}
          \end{array}
        \right)
      }
      \trans{sdelegs}{\Gamma}
      {
        \left(
          \begin{array}{l}
            \var{sds'}\\
            \var{eks'}
          \end{array}
        \right)
      }
      &
      {
        \left(
          \begin{array}{l}
            \var{dms}\\
            \var{dws}
          \end{array}
        \right)
      }
      \trans{adelegs}{[.., s] \restrictdom \var{sds'}}
      {
        \left(
          \begin{array}{l}
            \var{dms'}\\
            \var{dws'}
          \end{array}
        \right)
      }
    }
    {
      {\begin{array}{l}
         \mathcal{K} \\
         e\\
         s\\
         d
      \end{array}}
      \vdash
      {
        \left(
          \begin{array}{l}
            \var{dms}\\
            \var{dws}\\
            \var{sds}\\
            \var{eks}
          \end{array}
        \right)
      }
      \trans{deleg}{\Gamma}
      {
        \left(
          \begin{array}{l}
            \var{dms'}\\
            \var{dws'}\\
            \var{[s-d,~ s+d]} \restrictdom \var{sds'}\\
            \var{[e, ..]} \restrictdom \var{eks'}
          \end{array}
        \right)
      }
    }
  \end{equation}
  \caption{Delegation interface rules}
  \label{fig:rules:delegation-interface}
\end{figure}

\subsubsection{Delegation interface functions}
\label{sec:delegation-interface-functions}

\begin{figure}[htb]
  \begin{align*}
    & \fun{delegates} \in \DIState \to (\VKeyGen \times \VKey) \to \Bool & \text{delegation relationship}\\
    & \fun{delegates}~s~(\var{vk_s}, \var{vk_d}) = \var{vk_s} \mapsto \var{vk_d} \in (\var{dms}~s)\\
    \nextdef
    & \fun{initialDS} \in \VKeyGen \to \DIState & \text{initial delegation state}\\
    & \fun{initialDS}~\var{ks} =
      \left(
      \begin{array}{l}
        \var{ks} \mapsto \var{ks}\\
        \emptyset\\
        \epsilon\\
        \emptyset\\
      \end{array}
      \right)
  \end{align*}
  \caption{Delegation interface functions}
\end{figure}

\clearpage

\subsection{Update-proposals interface}
\label{sec:update-proposals-interface}

Figure~\ref{fig:ts-types:upi} defines the types of the transition systems
related with the update-proposals interface. The acronyms in the transition
labels have the following meaning:
\begin{description}
\item[UPIREG] Update-proposal registration.
\item[UPIVOTE] Update-proposal vote.
\item[UPIEND] Update-proposal endorsement.
\item[UPIEC] Update-proposal epoch-change.
\end{description}

\begin{figure}[htb]
  \emph{Update-proposals interface environments}
  \begin{align*}
    & \UPIEnv
      = \left(
      \begin{array}{r@{~\in~}lr}
        \var{k} & \mathbb{N} & \text{chain stability parameter}\\
        \var{u} & \mathbb{N} & \text{update proposals time-to-live}\\
        \var{e_c} & \Epoch & \text{current epoch}\\
        \var{b_n} & \BkNr & \text{current block number}\\
        \var{dms} & \VKeyGen \mapsto \VKey & \text{delegation map}\\
      \end{array}\right)
  \end{align*}
  %
  \emph{Update-proposals interface states}
  \begin{align*}
    & \UPIState = \\
    & \left(
      \begin{array}{r@{~\in~}lr}
        \var{e_p} & \Epoch & \text{previously seen epoch}\\
        (\var{pv}, \var{pps}) & \ProtVer \times \PPMMap
        & \text{current protocol information}\\
        (\var{pv_c}, \var{pps_c}) & \ProtVer \times \PPMMap
        & \text{candidate protocol information}\\
        \var{avs} & \ApName \mapsto (\mathbb{N} \times \BkNr)
        & \text{application versions}\\
        \var{rups} & \UPropId \mapsto (\ProtVer \times \PPMMap)
        & \text{registered update proposals}\\
        \var{raus} & \UPropId \mapsto (\ApName \times \mathbb{N})
        & \text{registered software update proposals}\\
        \var{eps} & \powerset{(\Epoch \times \VKeyGen)}
        & \text{proposals per-epoch per-key}\\
        \var{cps} & \UPropId \mapsto \BkNr & \text{confirmed proposals}\\
        \var{vts} & \powerset{(\UPropId \times \VKeyGen)} & \text{proposals votes}\\
        \var{bvs} & \powerset{(\ProtVer \times \VKeyGen)}
                           & \text{endorsement-key pairs}\\
        \var{pws} & \UPropId \mapsto \BkNr & \text{proposal timestamps}
      \end{array}\right)\\
  \end{align*}
  %
  \emph{Update-proposals interface transitions}
  \begin{equation*}
    \begin{array}{r@{~\in~}l}
      \_ \vdash \_ \trans{upireg}{\_} \_ &
      \powerset (\UPIEnv \times \UPIState \times \UProp \times \UPIState)\\
      \_ \vdash \_ \trans{upivote}{\_} \_ &
      \powerset (\UPIEnv \times \UPIState \times \Vote \times \UPIState)\\
      \_ \vdash \_ \trans{upiend}{\_} \_ &
      \powerset (\UPIEnv \times \UPIState
      \times (\ProtVer \times \VKey) \times \UPIState)\\
      \_ \vdash \_ \trans{upiec}{\_} \_ &
      \powerset (\UPIEnv \times \UPIState \times \Epoch \times \UPIState)
    \end{array}
  \end{equation*}
  \caption{Update-proposals interface transition-system types}
  \label{fig:ts-types:upi}
\end{figure}

\begin{figure}[htb]
  \begin{equation}
    \label{eq:rule:upi-reg-interface}
    \inference
    {
      {
        \begin{array}{l}
          \var{pv}\\
          \var{pps}\\
          \var{e_c}\\
          \var{avs}\\
          \var{dms}
        \end{array}
      }
      \vdash
      {
        \left(
          \begin{array}{l}
            \var{rups}\\
            \var{raus}\\
            \var{eps}
          \end{array}
        \right)
      }
      \trans{upreg}{\var{up}}
      {
        \left(
          \begin{array}{l}
            \var{rups'}\\
            \var{raus'}\\
            \var{eps'}
          \end{array}
        \right)
      }
      &
      pws' = pws \unionoverride \{ \upId{up} \mapsto b_n\}
    }
    {
      {
        \begin{array}{l}
          k\\
          u\\
          e_c\\
          b_n\\
          \var{dms}
        \end{array}
      }
      \vdash
      {
        \left(
          \begin{array}{l}
            \var{e_p}\\
            (\var{pv}, \var{pps})\\
            (\var{pv_c}, \var{pps_c})\\
            \var{avs}\\
            \var{rups}\\
            \var{raus}\\
            \var{eps}\\
            \var{cps}\\
            \var{vts}\\
            \var{bvs}\\
            \var{pws}
          \end{array}
        \right)
      }
      \trans{upireg}{\var{up}}
      {
        \left(
          \begin{array}{l}
            \var{e_p}\\
            (\var{pv}, \var{pps})\\
            (\var{pv_c}, \var{pps_c})\\
            \var{avs}\\
            \var{rups'}\\
            \var{raus'}\\
            \var{eps'}\\
            \var{cps}\\
            \var{vts}\\
            \var{bvs}\\
            \var{pws'}
          \end{array}
        \right)
      }
    }
  \end{equation}
  \caption{Update-proposals registration rules}
  \label{fig:rules:upi-reg-interface}
\end{figure}

\clearpage

Rule~\ref{eq:rule:upi-vote} models the effect of voting on an update proposal:
after a vote, a proposal might get confirmed, which means that it will be added
to the set $\var{cps'}$. In such case, the mapping of application names to
their latest version known to the ledger will be updated to include the
information in the confirmed proposal. Note that, unlike protocol updates,
software updates take effect as soon as a proposal is confirmed. In this rule,
we also delete the confirmed proposal id's from the set of registered
application update proposals ($\var{raus}$), since this information is no
longer needed once the application-name to software-version map ($\var{avs}$) is
updated.

\begin{figure}[htb]
  \begin{equation}
    \label{eq:rule:upi-vote}
    \inference
    {
      {
        \begin{array}{l}
          b_n\\
          \var{pps}\\
          \var{rups}\\
          \var{dms}
        \end{array}
      }
      \vdash
      {
        \left(
          \begin{array}{l}
            \var{cps}\\
            \var{vts}
          \end{array}
        \right)
      }
      \trans{upvote}{\var{v}}
      {
        \left(
          \begin{array}{l}
            \var{cps'}\\
            \var{vts'}
          \end{array}
        \right)
      }\\
      \var{avs_{new}} = \{ \var{an} \mapsto (\var{av}, \var{b_n})
      \mid \var{pid} \mapsto (\var{an}, \var{av}) \in \var{raups}
      ,~ \var{pid} \in \dom~\var{cps}
      \}
    }
    {
      {
        \begin{array}{l}
          k\\
          u\\
          e_c\\
          b_n\\
          \var{dms}
        \end{array}
      }
      \vdash
      {
        \left(
          \begin{array}{l}
            \var{e_p}\\
            (\var{pv}, \var{pps})\\
            (\var{pv_c}, \var{pps_c})\\
            \var{avs}\\
            \var{rups}\\
            \var{raus}\\
            \var{eps}\\
            \var{cps}\\
            \var{vts}\\
            \var{bvs}\\
            \var{pws}
          \end{array}
        \right)
      }
      \trans{upivote}{\var{v}}
      {
        \left(
          \begin{array}{l}
            \var{e_p}\\
            (\var{pv}, \var{pps})\\
            (\var{pv_c}, \var{pps_c})\\
            \var{avs} \unionoverride \var{avs_{new}}\\
            \var{rups}\\
            \dom~ \var{cps} \subtractdom \var{raus}\\
            \var{eps}\\
            \var{cps'}\\
            \var{vts'}\\
            \var{bvs}\\
            \var{pws}
          \end{array}
        \right)
      }
    }
  \end{equation}
  \caption{Voting on update-proposals rules}
  \label{fig:rules:upi-vote}
\end{figure}

\clearpage

The interface rule for protocol-block-version registration makes use of the
$\trans{upend}{}$ transition, passing only the portion of the state relevant
to it. In addition, the unconfirmed proposals that are older than $u$ blocks
are removed from the parts of the state that hold:
\begin{itemize}
\item the registered proposals,
\item the votes associated with the proposals,
\item the set of endorsement-key pairs, and
\item the block number in which proposals where added.
\end{itemize}
The epoch-key pairs set ($\var{eps}$) that represents which stakeholder
proposed an update in which epoch, is cleaned up in the rules given in
\cref{fig:rules:upi-ec}, which deals with epoch change.

In Rule~\ref{eq:rule:upi-pend}, the set of proposal id's $\var{pid_{keep}}$
contains only those proposals that haven't expired yet or that are confirmed.
Once a proposal $\var{up}$ is confirmed, it is removed from the set of
confirmed proposals ($\var{cps}$) either when it is adopted or when a protocol
version higher than that of $\var{up}$ gets adopted (see
Rule~\ref{eq:rule:upi-ec}).
%
The set of endorsement-key pairs is cleaned here as well as in the epoch change
rule (Rule~\ref{eq:rule:upi-ec}). The reason for this is that this set grows at
each block, and it can get considerably large if no proposal gets adopted at
the end on an epoch.

\begin{figure}[htb]
  \begin{equation}
    \label{eq:rule:upi-pend}
    \inference
    {
      {
        \begin{array}{l}
          k\\
          b_n\\
          (\var{pv}, \var{pps})\\
          \var{dms}\\
          \var{cps}\\
          \var{rups}
        \end{array}
      }
      \vdash
      {
        \left(
          \begin{array}{l}
            (\var{pv_c}, \var{pps_c})\\
            \var{bvs}
          \end{array}
        \right)
      }
      \trans{upend}{(\var{bv}, \var{vk})}
      {
        \left(
          \begin{array}{l}
            (\var{pv_c'}, \var{pps_c'})\\
            \var{bvs'}
          \end{array}
        \right)
      }\\
      {
        \begin{array}{r@{~=~}l}
          \var{pids_{keep}} & \dom~(pws \restrictrange [b_n - u, ..]) \cup \dom~\var{cps}\\
          \var{vs_{keep}} & \dom~(\range~\var{rups'})\\
          \var{rups'} & \var{pids_{keep}} \restrictdom \var{rups}
        \end{array}
      }
    }
    {
      {
        \begin{array}{l}
          k\\
          u\\
          e_c\\
          b_n\\
          \var{dms}
        \end{array}
      }
      \vdash
      {
        \left(
          \begin{array}{l}
            \var{e_p}\\
            (\var{pv}, \var{pps})\\
            (\var{pv_c}, \var{pps_c})\\
            \var{avs}\\
            \var{rups}\\
            \var{raus}\\
            \var{eps}\\
            \var{cps}\\
            \var{vts}\\
            \var{bvs}\\
            \var{pws}
          \end{array}
        \right)
      }
      \trans{upiend}{(\var{bv}, \var{vk})}
      {
        \left(
          \begin{array}{l}
            \var{e_p}\\
            (\var{pv}, \var{pps})\\
            (\var{pv_c'}, \var{pps_c'})\\
            \var{avs}\\
            \var{rups'}\\
            \var{pids_{keep}} \restrictdom \var{raus}\\
            \var{eps}\\
            \var{cps}\\
            \var{pids_{keep}} \restrictdom \var{vts}\\
            \var{vs_{keep}}  \restrictdom \var{bvs}\\
            \var{pids_{keep}} \restrictdom \var{pws}
          \end{array}
        \right)
      }
    }
  \end{equation}
  \caption{Proposal endorsement rules}
  \label{fig:rules:upi-pend}
\end{figure}

\clearpage

Rule~\ref{eq:rule:upi-ec} models how the protocol version and its parameters
are changed depending on the epoch in the signal ($e_n$ in this case). A change
only occurs if the new epoch is greater than the previously seen epoch ($e_p$).
In such case, the current epoch is also updated accordingly. The candidate
versions and parameters map $(\var{pv_c}, \var{pps_c})$ remain constant in this
transition, even if they become adopted. Note that, in the final state, we use
union override to define the updated parameters
($\var{pps} \unionoverride \var{pps'}$). This is because candidate proposal
might only update some parameters of the protocol.

Rule~\ref{eq:rule:upi-ec} performs cleanup of several state variables:
\begin{itemize}
\item We keep only the information of proposals whose proposed
  protocol-versions are greater than the newly adopted protocol version
  ($\var{pv'}$). This means that active proposals will be discarded even if the
  voting period is not over (it makes no sense to vote on a proposal that
  proposes to upgrade to an older version of the protocol).
\item The set $\var{eps}$ is updated so that only keys that proposed in epochs
  no earlier than $e_n$ are kept.
\item The registered software-update proposals need not be cleaned here, since
  this is done either when a proposal gets confirmed or when it expires.
\end{itemize}

\begin{figure}[htb]
  \begin{equation}
    \label{eq:rule:pvbump-change-noop}
    \inference
    {
      e_n \leq e_p
    }
    {
      (\var{pv_c}, \var{pps_c})
      \vdash
      {
        \left(
          \begin{array}{l}
            \var{e_p}\\
            \var{(\var{pv}, \var{pps})}\\
          \end{array}
        \right)
      }
      \trans{pvbump}{\var{e_n}}
      {
        \left(
          \begin{array}{l}
            \var{e_p}\\
            \var{(\var{pv}, \var{pps})}\\
          \end{array}
        \right)
      }
    }
  \end{equation}
  \nextdef
  \begin{equation}
    \label{eq:rule:pvbump-change}
    \inference
    {
      e_p < e_n
    }
    {
      (\var{pv_c}, \var{pps_c})
      \vdash
      {
        \left(
          \begin{array}{l}
            \var{e_p}\\
            \var{(\var{pv}, \var{pps})}
          \end{array}
        \right)
      }
      \trans{pvbump}{\var{e_n}}
      {
        \left(
          \begin{array}{l}
            \var{e_n}\\
            \var{(\var{pv_c}, \var{pps_c})}
          \end{array}
        \right)
      }
    }
  \end{equation}
  \nextdef
  \begin{equation}
    \label{eq:rule:upi-ec}
    \inference
    {
      (\var{pv_c}, \var{pps_c})
      \vdash
      {
        \left(
          \begin{array}{l}
            \var{e_p}\\
            \var{(\var{pv}, \var{pps})}
          \end{array}
        \right)
      }
      \trans{pvbump}{\var{e_n}}
      {
        \left(
          \begin{array}{l}
            \var{e'}\\
            \var{(\var{pv'}, \var{pps'})}\\
          \end{array}
        \right)
      }\\ ~ \\
      \var{pids_{keep}} = \{ \var{pid} \mid
      \var{pid} \mapsto (\var{pv_i}, \wcard) \in \var{rups}
      ,~ \var{pv'} \leq \var{pv_i}\}
    }
    {
      {
        \begin{array}{l}
          k\\
          u\\
          e_c\\
          b_n\\
          \var{dms}\\
        \end{array}
      }
      \vdash
      {
        \left(
          \begin{array}{l}
            \var{e_p}\\
            \var{(\var{pv}, \var{pps})}\\
            \var{(\var{pv_c}, \var{pps_c})}\\
            \var{avs}\\
            \var{rups}\\
            \var{raus}\\
            \var{eps}\\
            \var{cps}\\
            \var{vts}\\
            \var{bvs}\\
            \var{pws}
          \end{array}
        \right)
      }
      \trans{upiec}{\var{e_n}}
      {
        \left(
          \begin{array}{l}
            \var{e'}\\
            \var{(\var{pv'}, \var{pps} \unionoverride \var{pps'})}\\
            \var{(\var{pv_c}, \var{pps_c})}\\
            \var{avs}\\
            \var{pids_{keep}} \restrictdom \var{rups}\\
            \var{raus}\\
            {[e_n, ..]} \restrictdom \var{eps}\\
            \var{pids_{keep}} \restrictdom \var{cps}\\
            \var{pids_{keep}} \restrictdom \var{vts}\\
            {[\var{pv'}, ..]} \restrictdom \var{bvs}\\
            \var{pids_{keep}} \restrictdom \var{pws}
          \end{array}
        \right)
      }
    }
  \end{equation}
  \caption{Block version adoption on epoch change rules}
  \label{fig:rules:upi-ec}
\end{figure}
