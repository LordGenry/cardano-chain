\newcommand{\UProp}{\ensuremath{\type{UProp}}}
\newcommand{\UPropId}{\ensuremath{\type{UpId}}}
\newcommand{\UPropSD}{\ensuremath{\type{UpSD}}}
\newcommand{\ProtVer}{\ensuremath{\type{ProtVer}}}
\newcommand{\ProtPm}{\ensuremath{\type{Ppm}}}
\newcommand{\PPMMap}{\ensuremath{\type{PpmMap}}}
\newcommand{\UPVEnv}{\ensuremath{\type{UPVEnv}}}
\newcommand{\UPVState}{\ensuremath{\type{UPVState}}}
\newcommand{\UPLEnv}{\ensuremath{\type{UPLEnv}}}
\newcommand{\UPLState}{\ensuremath{\type{UPLState}}}
\newcommand{\UPAEnv}{\ensuremath{\type{UPAEnv}}}
\newcommand{\UPRState}{\ensuremath{\type{UPRState}}}
\newcommand{\Vote}{\ensuremath{\type{Vote}}}
\newcommand{\VEnv}{\ensuremath{\type{VEnv}}}
\newcommand{\VState}{\ensuremath{\type{VState}}}
\newcommand{\BVREnv}{\ensuremath{\type{BVREnv}}}
\newcommand{\BVRState}{\ensuremath{\type{BVRState}}}
\newcommand{\BkNr}{\ensuremath{\type{BkNr}}}
\newcommand{\ApName}{\ensuremath{\type{ApName}}}
\newcommand{\SWVer}{\ensuremath{\type{SWVer}}}

\newcommand{\upSize}[1]{\ensuremath{\fun{upSize}~\var{#1}}}
\newcommand{\upPV}[1]{\ensuremath{\fun{upPV}~\var{#1}}}
\newcommand{\upId}[1]{\ensuremath{\fun{upId}~\var{#1}}}
\newcommand{\upSig}[1]{\ensuremath{\fun{upSig}~\var{#1}}}
\newcommand{\upSigData}[1]{\ensuremath{\fun{upSigData}~\var{#1}}}
\newcommand{\upIssuer}[1]{\ensuremath{\fun{upIssuer}~\var{#1}}}
\newcommand{\upParams}[1]{\ensuremath{\fun{upParams}~\var{#1}}}
\newcommand{\upSwVer}[1]{\ensuremath{\fun{upSwVer}~\var{#1}}}
\newcommand{\vCaster}[1]{\ensuremath{\fun{vCaster}~\var{#1}}}
\newcommand{\vPropId}[1]{\ensuremath{\fun{vPropId}~\var{#1}}}
\newcommand{\vSig}[1]{\ensuremath{\fun{vSig}~\var{#1}}}

\lstset{ frame=tb,
       , language=Haskell
       , basicstyle=\footnotesize\ttfamily,
       , keywordstyle=\color{blue!80},
       , commentstyle=\itshape\color{purple!40!black},
       , identifierstyle=\bfseries\color{green!40!black},
       , stringstyle=\color{orange},
       }

\lstMakeShortInline[columns=fixed]`

\section{Update mechanism}
\label{sec:update}

This section formalizes the update mechanism by which the protocol parameters
get updated. This formalization is a simplification of the current update
mechanism implemented in
\href{https://github.com/input-output-hk/cardano-sl/}{\texttt{cardano-sl}}, and
partially documented in:
\begin{itemize}
\item \href{https://cardanodocs.com/technical/updater/}{Updater implementation}
\item \href{https://cardanodocs.com/cardano/update-mechanism/}{Update mechanism}
\item \href{https://github.com/input-output-hk/cardano-sl/blob/develop/wallet-new/docs/updates.md}{Updates technical documentation}
\item
  \href{https://github.com/input-output-hk/cardano-sl/blob/develop/docs/block-processing/us.md}{Update
    system consensus rules}
\end{itemize}

The reason for formalizing a simplified version of the current implementation
is that research work on blockchain update mechanisms is needed before
introducing a more complex update logic. Since this specification is to be
implemented in a federated setting, some of the constraints put in place in the
current implementation are no longer relevant. Once the research work is ready,
this specification can be extended to incorporate the research results.

\subsection{Update proposals}
\label{sec:update-proposals}

\begin{figure}[htb]
  \emph{Abstract types}
  %
  \begin{equation*}
    \begin{array}{r@{~\in~}lr}
      \var{up} & \UProp & \text{(protocol) update proposal}\\
      \var{p} & \ProtPm & \text{protocol parameters}\\
      \var{upd} & \type{UpdData} & \text{update data}\\
      \var{upa} & \type{UpdAttr} & \text{update attributes}\\
      \var{an} & \ApName & \text{application name}
    \end{array}
  \end{equation*}
  %
  \emph{Derived types}
  \begin{equation*}
    \begin{array}{r@{~\in~}l@{~=~}r@{~\in~}lr}
      \var{b_n} & \BkNr & n & \mathbb{N} & \text{block number}\\
      \var{pv} & \ProtVer & (\var{maj}, \var{min}, \var{alt})
      & (\mathbb{N}, \mathbb{N}, \mathbb{N}) & \text{protocol version}\\
      \var{pps} & \PPMMap & \var{pps} & \ProtPm \mapsto \Value
                                             & \text{protocol parameters map}\\
      \var{swv} & \SWVer
      & (\var{an}, \var{av}) & \ApName \times \mathbb{N}
      & \text{software version}\\
      \var{pb} & \UPropSD
      &
        {\left(\begin{array}{r l}
                 \var{pv}\\
                 \var{pps}\\
                 \var{swv}\\
                 \var{upd}\\
                 \var{upa}\\
               \end{array}\right)}
      & {
        \left(
        \begin{array}{l}
          \ProtVer\\
          \PPMMap\\
          \type{SWVer}\\
          \type{UpdData}\\
          \type{UpdAttr}\\
        \end{array}
                   \right)
                   }
               & \text{protocol update signed data}
    \end{array}
  \end{equation*}
  \emph{Abstract functions}
  %
  \begin{equation*}
    \begin{array}{r@{~\in~}lr}
      \fun{upIssuer} & \UProp \to \VKeyGen & \text{issuer of the update proposal}\\
      \fun{upSize} & \UProp \to \mathbb{N} & \text{update proposal size}\\
      \fun{upPV} & \UProp \to \ProtVer & \text{update proposal protocol version}\\
      \fun{upId} & \UProp \to \UPropId & \text{update proposal id}\\
      \fun{upParams} & \UProp \to \mathbb{\PPMMap}
                                           & \text{new parameters that are proposed}\\
      \fun{upSwVer} & \UProp \to \SWVer & \text{software version-update proposal}\\
      \fun{upSig} & \UProp \to \Sig & \text{update proposal signature}\\
      \fun{upSigdata} & \UProp \to \UPropSD & \text{update proposal signed data}\\
    \end{array}
  \end{equation*}
  \caption{Update proposals definitions}
  \label{fig:defs:update-proposals}
\end{figure}

The set of protocol parameters is assumed to contain the following keys, which
correspond with fields of the current `BlockVersionData` structure:
\begin{itemize}
\item Slot duration: $\var{slotDuration}$
\item Maximum block size: $\var{maxBlockSize}$
\item Maximum transaction size: $\var{maxTxSize}$
\item Maximum header size: $\var{maxHeaderSize}$
\item Maximum transaction size: $\var{maxTxSize}$
\item Maximum proposal size: $\var{maxProposalSize}$
\item Transaction fee policy: $\var{txFeePolicy}$
\item Script version: $\var{scriptVersion}$
\item Update adoption threshold: $\var{upAdptThd}$. This would correspond with
  `srInitThd` in `SoftForkRule`.
\end{itemize}
In addition we make use of the $\var{cfmThd}$ key, which determines the portion
of the stakeholders that have to vote for a proposal to consider it confirmed.
This key does not seem to have a corresponding field in `BlockVersionData`, so
to be consistent with the existing chain the value corresponding to
$\var{cfmThd}$ can be set to $4$, which represents the majority of the $7$
genesis keys.

\subsection{Update proposals registration}
\label{sec:update-proposals-registration}

First we model the validity of a proposal.

\begin{figure}[htb]
  \emph{Update proposals validity environments}
  \begin{equation*}
    \UPVEnv =
    \left(
      \begin{array}{r@{~\in~}lr}
        \var{pv} & \ProtVer & \text{adopted (current) protocol version}\\
        \var{pps} & \PPMMap & \text{adopted protocol parameters map}\\
        \var{avs} & \ApName \mapsto (\mathbb{N} \times \BkNr)
        & \text{application versions}\\
      \end{array}
    \right)
  \end{equation*}
  %
  \emph{Update proposals validity states}
  \begin{equation*}
    \UPVState
    = \left(
      \begin{array}{r@{~\in~}lr}
        \var{rups} & \UPropId \mapsto (\ProtVer \times \PPMMap)
        & \text{registered update proposals}\\
        \var{raus} & \UPropId \mapsto (\ApName \times \mathbb{N})
        & \text{registered software update proposals}\\
      \end{array}
    \right)
  \end{equation*}
  %
  \emph{Update proposals validity transitions}
    \begin{equation*}
    \var{\_} \vdash
    \var{\_} \trans{upv}{\_} \var{\_}
    \subseteq \powerset (\UPVEnv \times \UPVState \times \UProp \times \UPVState)
  \end{equation*}
  \caption{Update proposals validity transition-system types}
  \label{fig:ts-types:up-validity}
\end{figure}

In Rule~\ref{eq:rule:up-validity} we see that a new proposal:
\begin{itemize}
\item if it changes the protocol version, it must do it in a consistent manner:
  \begin{itemize}
  \item The proposed version must be lexicographically bigger or equal than the
    current version.
  \item The major versions of the proposed and current version must differ in
    at most one.
  \item If the proposed major version is equal to the current major
    version, then the proposed minor version must be incremented by one.
  \item If the proposed major version is larger than the current major version,
    then the proposed minor version must be zero.
  \end{itemize}
\item must be consistent with the current protocol parameters:
  \begin{itemize}
  \item the proposal size must not exceed the maximum size specified by
    the current protocol parameters,
  \item the proposed new maximum block size should be not greater than twice
    current maximum block size,
  \item the maximum transaction size must be smaller than the maximum block
    size (this requirement is \textbf{crucial} for having every transaction
    fitting in a block \footnote{TODO: we should discuss if this is enough,
      since a block might contain other data that contributes to its total
      size}), and
  \item the proposed new script version can be incremented by at most 1.
  \end{itemize}
\item must be new:
  \begin{itemize}
  \item must not exist in the set of registered proposals, and
  \item must have a unique version among the current active proposals. This
    implies that a proposal is uniquely determined by the protocol version it
    proposes.
  \end{itemize}
\item must increase the software version by at most one (see
  \cref{eq:func:av-can-follow}).
\item must not contain a software update proposal that was already registered
  (in $\var{raups}$).
\item the protocol version and parameters, along with some other data (see the
  definition of $\fun{upSigdata}$), must be signed by the proposal issuer.
\end{itemize}

\begin{figure}[htb]
  \begin{equation}
    \label{eq:func:pv-can-follow}
    \begin{array}{r c l}
      \fun{pvCanFollow}~(\var{mj_n}, \var{mi_n}, \var{a_n})~(\var{mj_p}, \var{mi_p}, \var{a_p})
      & = & (\var{mj_p}, \var{mi_p}, \var{a_p}) \leq (\var{mj_n}, \var{mi_n}, \var{a_n})\\
      & \wedge & 0 \leq \var{mj_n} - \var{mj_p} \leq 1\\
      & \wedge & (\var{mj_p} = \var{mj_n} \Rightarrow \var{mi_p} + 1 = \var{mi_n}))\\
      & \wedge & (\var{mj_p} + 1 = \var{mj_n} \Rightarrow \var{mi_n} = 0)
    \end{array}
  \end{equation}
  \nextdef
  \begin{equation}
    \label{eq:func:can-update}
    \begin{array}{l}
      \fun{canUpdate}~\var{pps}~\var{up}\\
      {\begin{array}{r c l}
         & = & \var{maxProposalSize} \mapsto \var{mus} \in \var{pps}\\
         & \wedge & \upSize{up} \leq \var{mus}\\
         & \wedge & (\var{maxBlockSize} \mapsto \var{bszm_{up}} \in (\upParams{up})\\
         & & \Rightarrow \var{maxBlockSize} \mapsto \var{bszm} \in \var{pps}
             \wedge  \var{bszm_{up}} \leq 2\var{bszm})\\
         & \wedge & (\var{maxBlockSize} \mapsto \var{bszm_{up}} \in (\upParams{up})
         \wedge \var{maxTxSize} \mapsto \var{txzm_{up}} \in (\upParams{up})\\
         & & \Rightarrow \var{txzm_{up}} < \var{bszm_{up}})\\
         & \wedge & (\var{scriptVersion} \mapsto \var{sv_{up}} \in (\upParams{up}) \\
         & & \Rightarrow \var{scriptVersion} \mapsto \var{sv} \in \var{pps}
             \wedge  0 \leq \var{sv_{up}} - \var{sv} \leq 1)
       \end{array}}
    \end{array}
  \end{equation}
  \nextdef
  \begin{equation}
    \label{eq:func:pv-is-new}
    \begin{array}{r c l}
      \fun{pvIsNew}~\var{rups}~(\var{pid}, \var{nv})
      & = &  \var{pid} \notin \dom \var{rups}
            \wedge \var{nv} \notin \dom (\range \var{rups})
    \end{array}
  \end{equation}
  \nextdef
  \begin{equation}
    \label{eq:func:av-can-follow}
    \begin{array}{r c l}
      \fun{avCanFollow}~\var{avs}~(\var{an}, \var{av}) & =
      & (\var{an} \mapsto (\var{av_c}, \wcard) \in \var{avs}
        \Rightarrow 0 \leq \var{av} - \var{av_c} \leq 1)\\
      & \vee & (\var{an} \notin \dom~\var{avs} \Rightarrow \var{av} = 0)
    \end{array}
  \end{equation}
  \caption{Update validity functions}
\end{figure}

\begin{figure}[htb]
  \begin{equation}
    \label{eq:rule:up-av-validity}
    \inference
    {
      (\var{an}, \var{av}) = \upSwVer{up}
      & \fun{avCanFollow}~\var{avs}~(\var{an}, \var{av})
      & (\var{an}, \var{av}) \notin \range~\var{raus}
    }
    {
      {
        \begin{array}{l}
          \var{avs}
        \end{array}
      }
      \vdash
      {
        \left(
          \begin{array}{l}
            \var{raus}
          \end{array}
        \right)
      }
      \trans{upsvv}{up}
      {
        \left(
          \begin{array}{l}
            \var{raus} \unionoverride \{ \upId{up} \mapsto (\var{an}, \var{av})\}
          \end{array}
        \right)
      }
    }
  \end{equation}
  \nextdef
    \begin{equation}
    \label{eq:rule:up-pv-validity}
    \inference
    {
      \var{vk} = \upIssuer{up}
      & \var{pid} = \upId{up}
      & \var{nv} = \upPV{up}
      & \var{pps_n} = \upParams{up}\\
      & \fun{pvCanFollow}~\var{nv}~\var{pv}
      & \fun{canUpdate}~\var{pps}~\var{up}
      & \fun{pvIsNew}~\var{rups}~(\var{pid}, \var{nv})
    }
    {
      {
        \begin{array}{l}
          \var{pv}\\
          \var{pps}
        \end{array}
      }
      \vdash
      {
        \left(
          \begin{array}{l}
            \var{rups}
          \end{array}
        \right)
      }
      \trans{uppvv}{\var{up}}
      {
        \left(
          \begin{array}{l}
            \var{rups} \unionoverride \{ \var{pid} \mapsto (\var{nv}, \var{pps_n}) \}
          \end{array}
        \right)
      }
    }
  \end{equation}
  \nextdef
  \begin{equation}
    \label{eq:rule:up-validity}
    \inference
    {
      {
        \begin{array}{l}
          \var{pv}\\
          \var{pps}
        \end{array}
      }
      \vdash
      {
        \left(
          \begin{array}{l}
            \var{rups}
          \end{array}
        \right)
      }
      \trans{uppvv}{\var{up}}
      {
        \left(
          \begin{array}{l}
            \var{rups'}
          \end{array}
        \right)
      }
      &
      {
        \begin{array}{l}
          \var{avs}
        \end{array}
      }
      \vdash
      {
        \left(
          \begin{array}{l}
            \var{raus}
          \end{array}
        \right)
      }
      \trans{upsvv}{up}
      {
        \left(
          \begin{array}{l}
            \var{raus'}
          \end{array}
        \right)
      }
      \\
      \mathcal{V}_{\var{vk}}\serialised{\upSigData{up}}_{(\upSig{up})}
    }
    {
      {
        \begin{array}{l}
          \var{pv}\\
          \var{pps}\\
          \var{avs}
        \end{array}
      }
      \vdash
      {
        \left(
          \begin{array}{l}
            \var{rups}\\
            \var{raus}
          \end{array}
        \right)
      }
      \trans{upv}{\var{up}}
      {
        \left(
          \begin{array}{l}
            \var{rups'}\\
            \var{raus'}
          \end{array}
        \right)
      }
    }
  \end{equation}
  \caption{Update proposals validity rules}
  \label{fig:rules:up-validity}
\end{figure}

\clearpage

\begin{figure}[htb]
  \emph{Update proposals limits  environments}
    \begin{equation*}
    \UPLEnv =
    \left(
      \begin{array}{r@{~\in~}lr}
        \var{e_c} & \Epoch & \text{current epoch}\\
        \var{dms} & \VKeyGen \mapsto \VKey & \text{delegation map}\\
      \end{array}
    \right)
  \end{equation*}
  %
  \emph{Update proposals limits states}
  \begin{equation*}
    \UPLState
    = \left(
      \begin{array}{r@{~\in~}lr}
        \var{eps} & \powerset{(\Epoch \times \VKeyGen)} & \text{proposals per-epoch per-key}\\
      \end{array}
    \right)
  \end{equation*}
  %
  \emph{Update proposals limits transitions}
  \begin{equation*}
    \var{\_} \vdash
    \var{\_} \trans{upl}{\_} \var{\_}
    \subseteq \powerset (\UPLEnv \times \UPLState \times \UProp \times \UPLState)
  \end{equation*}
  \caption{Update proposals limits transition-system types}
  \label{fig:ts-types:up-limits}
\end{figure}

In Rule~\ref{eq:rule:up-limits}:
\begin{itemize}
\item We consider the update proposal issuers to be the delegators of the key
  ($\var{vk}$) that is associated with the proposal under consideration
  ($\var{up}$).

\item We check that the issuer of a proposal was delegated by a genesis key
  (which are in the domain of $\var{dms}$).
\item We check that no key that delegated to $\var{vk}$ has issued a proposal
  in the current epoch $\var{e_c}$.
\item If the above checks succeeds, then all the delegators of $\var{vk}$ are
  added to the set of keys that proposed in this epoch.
\end{itemize}

\begin{figure}[htb]
  \begin{equation}
    \label{eq:rule:up-limits}
    \inference
    {\var{vk} = \upIssuer{up}
      &  \var{dms} \restrictrange \{\var{vk}\} \neq \emptyset\\
      & \var{eps_{vk}} = \{ (e_c, \var{vk_s})
      \mid \var{vk_s} \mapsto \var{vk} \in \var{dms}\}
      & \var{eps_{vk}} \cap \var{eps} = \emptyset
    }
    {
      {\begin{array}{l}
         \var{e_c}\\
         \var{dms}
       \end{array}
      }
      \vdash
      {
        \left(
          \begin{array}{l}
            \var{eps}
          \end{array}
        \right)
      }
      \trans{upl}{\var{up}}
      {
        \left(
          \begin{array}{l}
            \var{eps} \cup \var{eps_{vk}}
          \end{array}
        \right)
      }
    }
  \end{equation}
  \caption{Update proposals limits rules}
  \label{fig:rules:up-limits}
\end{figure}

\begin{figure}[htb]
  \emph{Update proposals registration  environments}
    \begin{equation*}
    \UPAEnv =
    \left(
      \begin{array}{r@{~\in~}lr}
        \var{pv} & \ProtVer & \text{adopted (current) protocol version}\\
        \var{pps} & \PPMMap & \text{adopted protocol parameters map}\\
        \var{e_c} & \Epoch & \text{current epoch}\\
        \var{avs} & \ApName \mapsto (\mathbb{N} \times \BkNr)
        & \text{application versions}\\
        \var{dms} & \VKeyGen \mapsto \VKey & \text{delegation map}\\
      \end{array}
    \right)
  \end{equation*}
  %
  \emph{Update proposals registration states}
  \begin{align*}
    & \UPRState = \\
    & \left(
      \begin{array}{r@{~\in~}lr}
        \var{rups} & \UPropId \mapsto (\ProtVer \times \PPMMap)
        & \text{registered update proposals}\\
        \var{raus} & \UPropId \mapsto (\ApName \times \mathbb{N})
        & \text{registered software update proposals}\\
        \var{eps} & \powerset{(\Epoch \times \VKeyGen)} & \text{proposals per-epoch per-key}\\
      \end{array}
    \right)
  \end{align*}
  %
  \emph{Update proposals registration transitions}
  \begin{equation*}
    \var{\_} \vdash
    \var{\_} \trans{upreg}{\_} \var{\_}
    \subseteq \powerset (\UPAEnv \times \UPRState \times \UProp \times \UPRState)
  \end{equation*}
  \caption{Update proposals registration transition-system types}
  \label{fig:ts-types:up-registration}
\end{figure}

\begin{figure}[htb]
  \begin{equation}
    \label{eq:rule:up-registration}
    \inference
    {
      {
        \begin{array}{l}
          \var{pv}\\
          \var{pps}\\
          \var{avs}
        \end{array}
      }
      \vdash
      {
        \left(
          \begin{array}{l}
            \var{rups}\\
            \var{raus}\\
          \end{array}
        \right)
      }
      \trans{upv}{\var{up}}
      {
        \left(
          \begin{array}{l}
            \var{rups'}\\
            \var{raus'}\\
          \end{array}
        \right)
      }
      &
      {\begin{array}{l}
          \var{e_c}\\
          \var{dms}
        \end{array}
      }
      \vdash
      {
        \left(
          \begin{array}{l}
            \var{eps}
          \end{array}
        \right)
      }
      \trans{upl}{\var{up}}
      {
        \left(
          \begin{array}{l}
            \var{eps'}
          \end{array}
        \right)
      }
    }
    {
      {
        \begin{array}{l}
          \var{pv}\\
          \var{pps}\\
          \var{e_c}\\
          \var{avs}\\
          \var{dms}
        \end{array}
      }
      \vdash
      {
        \left(
          \begin{array}{l}
            \var{rups}\\
            \var{raus}\\
            \var{eps}
          \end{array}
        \right)
      }
      \trans{upreg}{\var{up}}
      {
        \left(
          \begin{array}{l}
            \var{rups'}\\
            \var{raus'}\\
            \var{eps'}
          \end{array}
        \right)
      }
    }
  \end{equation}
  \caption{Update registration rules}
  \label{fig:rules:up-registration}
\end{figure}

\clearpage

\subsection{Voting on update proposals}
\label{sec:voting-on-update-proposals}

\begin{figure}[htb]
  \emph{Abstract types}
  %
  \begin{equation*}
    \begin{array}{r@{~\in~}lr}
      \var{v} & \Vote & \text{vote on an update proposal}
    \end{array}
  \end{equation*}
  %
  \emph{Abstract functions}
  \begin{align*}
    & \fun{vCaster} \in \Vote \to \VKey & \text{caster of a vote}\\
    & \fun{vPropId} \in \Vote \to \UPropId & \text{proposal id that is being voted}\\
    & \fun{vSig} \in \Vote \to \Sig & \text{vote signature}
  \end{align*}
  \caption{Voting definitions}
  \label{fig:defs:voting}
\end{figure}

\begin{figure}[htb]
  \emph{Voting environments}
  \begin{align*}
    & \VEnv
      = \left(
      \begin{array}{r@{~\in~}lr}
        \var{rups} & \UPropId \mapsto (\ProtVer \times \PPMMap)
        & \text{registered update proposals}\\
        \var{dms} & \VKeyGen \mapsto \VKey & \text{delegation map}
      \end{array}\right)
  \end{align*}
  %
  \emph{Voting states}
  \begin{align*}
    & \VState
      = \left(
      \begin{array}{r@{~\in~}lr}
        \var{vts} & \powerset{(\UPropId \times \VKeyGen)} & \text{votes}
      \end{array}\right)
  \end{align*}
  %
  \emph{Voting transitions}
    \begin{equation*}
    \_ \vdash \_ \trans{addvote}{\_} \_ \in
    \powerset (\VEnv \times \VState \times \Vote \times \VState)
    \end{equation*}
  \caption{Voting transition-system types}
  \label{fig:ts-types:voting}
\end{figure}

In Rule~\ref{eq:rule:voting}:
\begin{itemize}
\item Only genesis keys can vote on an update proposal, although votes can be
  cast by delegates of these genesis keys.
\item We count one vote per genesis key that delegated to the key that is
  casting the vote.
\item The vote must refer to a registered update proposal.
\item The proposal id must be signed by the key that is casting the vote.
\item It is possible for the same genesis key to vote multiple times for
  the same proposal, however this vote will be counted once (note that we're
  taking the union of the key-proposal-id pairs).
\end{itemize}

\begin{figure}[htb]
  \begin{equation}
    \label{eq:rule:voting}
    \inference
    {
      \var{pid} = \vPropId{v} & \var{vk} = \vCaster{v} &
      \var{vts}_{\var{pid}} =
      \{ (\var{pid}, \var{vk_s}) \mid \var{vk_s} \mapsto \var{vk} \in \var{dms} \}\\
      & \var{pid} \in \dom \var{rups} &
      \mathcal{V}_{\var{vk}}\serialised{\var{pid}}_{(\vSig{v})}\\
    }
    {
      {
        \begin{array}{l}
          \var{rups}\\
          \var{dms}
        \end{array}
      }
      \vdash
      {
        \left(
          \begin{array}{l}
            \var{vts}
          \end{array}
        \right)
      }
      \trans{addvote}{\var{v}}
      {
        \left(
          \begin{array}{l}
            \var{vts} \cup \var{vts}_{\var{pid}}\\
          \end{array}
        \right)
      }
    }
  \end{equation}
  \caption{Update voting rules}
  \label{fig:rules:voting}
\end{figure}

\clearpage

\begin{figure}[htb]
  \emph{Vote registration environments}
  \begin{align*}
    & \VEnv
      = \left(
      \begin{array}{r@{~\in~}lr}
        \var{b_n} & \BkNr & \text{current block number}\\
        \var{pps} & \PPMMap & \text{current protocol parameters map}\\
        \var{rups} & \UPropId \mapsto (\ProtVer \times \PPMMap)
        & \text{registered update proposals}\\
        \var{dms} & \VKeyGen \mapsto \VKey & \text{delegation map}
      \end{array}\right)
  \end{align*}
  %
  \emph{Vote registration states}
  \begin{align*}
    & \VState
      = \left(
      \begin{array}{r@{~\in~}lr}
        \var{cps} & \UPropId \mapsto \BkNr & \text{confirmed proposals}\\
        \var{vts} & \powerset{(\UPropId \times \VKeyGen)} & \text{votes}
      \end{array}\right)
  \end{align*}
  %
  \emph{Vote registration transitions}
    \begin{equation*}
    \_ \vdash \_ \trans{UPVOTE}{\_} \_ \in
    \powerset (\VEnv \times \VState \times \Vote \times \VState)
    \end{equation*}
  \caption{Vote registration transition-system types}
  \label{fig:ts-types:vote-reg}
\end{figure}

The rules in Figure~\ref{fig:rules:up-vote-reg} model the registration of a vote:
\begin{itemize}
\item The vote gets added to the list set of votes per-proposal ($\var{vts}$),
  via transition $\trans{addvote}{}$.
\item If the number of votes for the proposal $v$ refers to exceeds the
  confirmation threshold and this proposal was not confirmed already, then the
  proposal gets added to the set of confirmed proposals ($\var{cps}$). The
  reason why we check that the proposal was not already confirmed, is that we
  want to keep in $\var{cps}$ the earliest block number in which the proposal
  was confirmed.
\end{itemize}

\begin{figure}[htb]
  \begin{equation}
    \label{eq:rule:up-no-confirmation}
    \inference
    {
      {
        \begin{array}{l}
          \var{rups}\\
          \var{dms}
        \end{array}
      }
      \vdash
      {
        \left(
          \begin{array}{l}
            \var{vts}
          \end{array}
        \right)
      }
      \trans{addvote}{\var{v}}
      {
        \left(
          \begin{array}{l}
            \var{vts'}
          \end{array}
        \right)
      }\\
      \var{pid} = \vPropId{v}
      & \var{cfmThd} \mapsto t \in \var{pps}
      & (\size{\{\var{pid}\} \restrictdom \var{vts'}} < t
      \vee \var{pid} \in \dom~\var{cps}
      )
    }
    {
      {
        \begin{array}{l}
          b_n\\
          \var{pps}\\
          \var{rups}\\
          \var{dms}
        \end{array}
      }
      \vdash
      {
        \left(
          \begin{array}{l}
            \var{cps}\\
            \var{vts}
          \end{array}
        \right)
      }
      \trans{upvote}{\var{v}}
      {
        \left(
          \begin{array}{l}
            \var{cps}\\
            \var{vts'}
          \end{array}
        \right)
      }
    }
  \end{equation}
  \nextdef
  \begin{equation}
    \label{eq:rule:up-vote-reg}
    \inference
    {
      {
        \begin{array}{l}
          \var{rups}\\
          \var{dms}
        \end{array}
      }
      \vdash
      {
        \left(
          \begin{array}{l}
            \var{vts}
          \end{array}
        \right)
      }
      \trans{addvote}{\var{v}}
      {
        \left(
          \begin{array}{l}
            \var{vts'}
          \end{array}
        \right)
      }\\
      \var{pid} = \vPropId{v}
      & \var{cfmThd} \mapsto t \in \var{pps}
      & t \leq \size{\{\var{pid}\} \restrictdom \var{vts'}}
    }
    {
      {
        \begin{array}{l}
          \var{b_n}\\
          \var{pps}\\
          \var{rups}\\
          \var{dms}
        \end{array}
      }
      \vdash
      {
        \left(
          \begin{array}{l}
            \var{cps}\\
            \var{vts}
          \end{array}
        \right)
      }
      \trans{upvote}{\var{v}}
      {
        \left(
          \begin{array}{l}
            \var{cps} \unionoverride  \{\var{pid} \mapsto b_n\} \\
            \var{vts'}
          \end{array}
        \right)
      }
    }
  \end{equation}
  \caption{Vote registration rules}
  \label{fig:rules:up-vote-reg}
\end{figure}

\clearpage

\subsection{Update-proposal endorsement}
\label{sec:proposal-endorsement}

Figure~\ref{fig:ts-types:up-end} shows the types of the transition system
associated with the registration of candidate protocol versions present in
blocks. Some clarifications are in order:
\begin{itemize}
\item The $k$ parameter is used to determined when a confirmed proposal is
  stable. Given we are in a current block $b_n$, all update proposals confirmed
  at or before block $b_n - k$ are deemed stable.
\item For the sake of conciseness, we omit the types associated to the
  transitions $\trans{addbvvk}{}$, since they can be inferred from the types of
  the $\trans{upend}{}$ transitions.
\end{itemize}

\begin{figure}[htb]
  \emph{Update-proposal endorsement environments}
  \begin{align*}
    & \BVREnv
      = \left(
      \begin{array}{r@{~\in~}lr}
        \var{k} & \mathbb{N} & \text{chain stability parameter}\\
        \var{b_n} & \BkNr & \text{current block number}\\
        (\var{pv}, \var{pps}) & \ProtVer \times \PPMMap
                             & \text{current protocol parameters map}\\
        \var{dms} & \VKeyGen \mapsto \VKey & \text{delegation map}\\
        \var{cps} & \UPropId \mapsto \BkNr & \text{confirmed proposals}\\
        \var{rups} & \UPropId \mapsto (\ProtVer \times \PPMMap)
                             & \text{registered update proposals}\\
      \end{array}\right)
  \end{align*}
  %
  \emph{Update-proposal endorsement states}
  \begin{align*}
    & \BVRState
      = \left(
      \begin{array}{r@{~\in~}lr}
        (\var{pv_c}, \var{pps_c}) & \ProtVer \times \PPMMap
        & \text{candidate protocol}\\
        \var{bvs} & \powerset{(\ProtVer \times \VKeyGen)}
        & \text{endorsement-key pairs}
      \end{array}\right)
  \end{align*}
  %
  \emph{Update-proposal endorsement transitions}
    \begin{equation*}
    \_ \vdash \_ \trans{upend}{\_} \_ \in
    \powerset (\BVREnv \times \BVRState
    \times (\ProtVer \times \VKey) \times \BVRState)
    \end{equation*}
  \caption{Update-proposal endorsement transition-system types}
  \label{fig:ts-types:up-end}
\end{figure}

Rules in \cref{fig:rules:up-end} specify what happens when a block issuer
signals that it is ready to upgrade to a new protocol version, given in the
rule by $\var{bv}$:
\begin{itemize}
\item The set $\var{bvs}$, containing which genesis keys are ready to adopt a
  given protocol version, is updated to reflect that all the delegators of the
  block issuer (identified by its verifying key, $\var{vk}$) are ready to
  upgrade to $\var{bv}$. Given a pair $(\var{pv}, ~\var{vk_s}) \in \var{bvs}$,
  we say that (the owner of) key $\var{vk_s}$ endorses the (proposed) protocol
  version $\var{pv}$.
\item If the candidate protocol version $\var{bv}$ is greater than the current
  candidate version $\var{pv_c}$, and there is a significant number of blocks
  signed with that version (condition formalized in
  \cref{eq:predicate:canadopt}), then the new candidate version becomes
  $\var{bv}$. Note that we only compare candidates block versions, and not the
  current protocol version $\var{pv}$. If this rule is used in an initial state
  in which $\var{pv} \leq \var{pv_c}$, then this invariant is maintained by
  these rules.
\item Endorsed protocol version $\var{bv}$ must refer to a registered update
  proposal (which are contained in $\var{rups}$), and this update proposal must
  have been confirmed at least $k$ blocks ago, to ensure stability of the
  confirmation. Note that this rule does not guarantee that a confirmed
  proposal will have $k$ blocks (or $2k$ blocks) before the end of the epoch
  where nodes can endorse this version. In such case, the proposal can only be
  adopted in the next epoch. Figure~\ref{fig:up-confirmed-too-late} shows
  an example of a proposal being confirmed too late in an epoch, where it is
  not possible to get the enough endorsements in the remaining window. In this
  Figure we take $k = 2$.
\end{itemize}

\begin{figure}[htb]
  \begin{equation}
    \label{eq:predicate:canadopt}
    \begin{array}{r c l}
      \fun{canAdopt}~\var{pps}~\var{bvs}~\var{bv}
      & =
      & \var{upAdptThd} \mapsto t \in pps \wedge
        t \leq \size{\{\var{bv}\} \restrictdom \var{bvs}}\\
    \end{array}
  \end{equation}
  \caption{Update-proposal endorsement functions}
\end{figure}

\begin{figure}[htb]
  \begin{equation}
    \label{eq:rule:addbvvk}
    \inference
    {
      \var{bvs'} = \var{bvs} \cup
      \{ (\var{bv}, \var{vk_s}) \mid \var{vk_s} \mapsto \var{vk} \in \var{dms} \}
    }
    {
      {
        \begin{array}{l}
          \var{dms}
        \end{array}
      }
      \vdash
      {
        \left(
          \begin{array}{l}
            bvs
          \end{array}
        \right)
      }
      \trans{addbvvk}{(\var{bv}, \var{vk})}
      {
        \left(
          \begin{array}{l}
            \var{bvs'}
          \end{array}
        \right)
      }
    }
  \end{equation}
  %
  \nextdef
  %
  \begin{equation}
    \label{eq:rule:up-adopted}
    \inference
    {
      \var{bv} \leq \var{pv_c}
      &
      {
        \begin{array}{l}
          \var{dms}
        \end{array}
      }
      \vdash
      {
        \left(
          \begin{array}{l}
            bvs
          \end{array}
        \right)
      }
      \trans{addbvvk}{(\var{bv}, \var{vk})}
      {
        \left(
          \begin{array}{l}
            bvs'
          \end{array}
        \right)
      }
    }
    {
      {
        \begin{array}{l}
          k\\
          b_n\\
          (\var{pv}, \var{pps})\\
          \var{dms}\\
          \var{cps}\\
          \var{rups}
        \end{array}
      }
      \vdash
      {
        \left(
          \begin{array}{l}
            (\var{pv_c}, \var{pps_c})\\
            \var{bvs}
          \end{array}
        \right)
      }
      \trans{upend}{(\var{bv}, \var{vk})}
      {
        \left(
          \begin{array}{l}
            (\var{pv_c}, \var{pps_c})\\
            \var{bvs'}
          \end{array}
        \right)
      }
    }
  \end{equation}
  %
  \nextdef
  %
  \begin{equation}
    \label{eq:rule:up-no-adoption}
    \inference
    {
      \var{pv_c} < \var{bv}
      &
      {
        \begin{array}{l}
          \var{dms}
        \end{array}
      }
      \vdash
      {
        \left(
          \begin{array}{l}
            bvs
          \end{array}
        \right)
      }
      \trans{addbvvk}{(\var{bv}, \var{vk})}
      {
        \left(
          \begin{array}{l}
            bvs'
          \end{array}
        \right)
      }
      & \neg (\fun{canAdopt}~\var{pps}~\var{bvs'}~\var{bv})\\
      \var{pid} \mapsto (\var{bv}, \wcard) \in \var{rups}
      & \var{pid} \in \dom~(\var{cps} \restrictrange [.., b_n - k])
    }
    {
      {
        \begin{array}{l}
          k\\
          b_n\\
          (\var{pv}, \var{pps})\\
          \var{dms}\\
          \var{cps}\\
          \var{rups}
        \end{array}
      }
      \vdash
      {
        \left(
          \begin{array}{l}
            (\var{pv_c}, \var{pps_c})\\
            \var{bvs}
          \end{array}
        \right)
      }
      \trans{upend}{(\var{bv}, \var{vk})}
      {
        \left(
          \begin{array}{l}
            (\var{pv_c}, \var{pps_c})\\
            \var{bvs'}
          \end{array}
        \right)
      }
    }
  \end{equation}
  %
  \nextdef
  %
  \begin{equation}
    \label{eq:rule:up-adoption}
    \inference
    {
      \var{pv_c} < \var{bv}
      &
      {
        \begin{array}{l}
          \var{dms}
        \end{array}
      }
      \vdash
      {
        \left(
          \begin{array}{l}
            bvs
          \end{array}
        \right)
      }
      \trans{addbvvk}{(\var{bv}, \var{vk})}
      {
        \left(
          \begin{array}{l}
            bvs'
          \end{array}
        \right)
      }
      &
      \fun{canAdopt}~\var{pps}~\var{bvs'}~\var{bv}\\
      \var{pid} \mapsto (\var{bv}, \var{pps_c'}) \in \var{rups}
      & \var{pid} \in \dom~(\var{cps} \restrictrange [.., b_n - k])\\
    }
    {
      {
        \begin{array}{l}
          k\\
          b_n\\
          (\var{pv}, \var{pps})\\
          \var{dms}\\
          \var{cps}\\
          \var{rups}
        \end{array}
      }
      \vdash
      {
        \left(
          \begin{array}{l}
            (\var{pv_c}, \var{pps_c})\\
            \var{bvs}
          \end{array}
        \right)
      }
      \trans{upend}{(\var{bv}, \var{vk})}
      {
        \left(
          \begin{array}{l}
            (\var{bv}, \var{pps_c'})\\
            \var{bvs'}
          \end{array}
        \right)
      }
    }
  \end{equation}
  \caption{Update-proposal endorsement rules}
  \label{fig:rules:up-end}
\end{figure}


\begin{figure}[htb]
  \centering
  \begin{tikzpicture}
    %%
    %% Macros used in this picture
    %%
    %
    % Number of slots
    \pgfmathsetmacro{\nrSlots}{11}
    % Slot in which the proposal gets confirmed
    \pgfmathsetmacro{\cSlot}{1}
    % Our special K.
    \pgfmathsetmacro{\K}{2}
    % Epoch end.
    \pgfmathsetmacro{\eend}{7}

    % Draw the horizontal line
    \draw[thick, -Triangle] (0,0) -- (\nrSlots,0)
    node[font=\scriptsize,below left=3pt and -8pt]{slots};

    % % draw vertical lines
    \foreach \x in {0,1,...,10}
    \draw (\x cm, 3pt) -- (\x cm, -3pt);

    % Add a label to with the block number in which the proposal got confirmed.
    \node at (\cSlot, -.7) {$b_j$};

    % Update in cps
    \node at (\cSlot, -1.5) {$\var{cps'} = \var{cps} \cup \{ \var{pid} \mapsto b_j \}$};

    % The no-endorsements red bar.
    \draw[red, line width=4pt] (\cSlot, .5) -- +(2*\K, 0);

    % Brace above the no-endorsement window bar.
    \draw[thick, red, decorate, decoration={brace, amplitude=5pt}]
    (\cSlot, .7) -- +(2*\K, 0)
    node[black!20!red, midway, above=4pt, font=\scriptsize] {No-endorsements possible};

    % The endorsements window.
    \coordinate (ewStart) at (\cSlot + 2*\K, .5);
    \coordinate (ewEnd) at ($(\eend, .5)$);
    \draw[green, line width=4pt]
    (ewStart) -- (ewEnd);

    % Brace above the endorsements window
    \coordinate (ewStartB) at ($(ewStart) + (0, 0.2)$);
    \coordinate (ewEndB) at ($(ewEnd) + (0, 0.2)$);
    \draw[thick, green, decorate, decoration={brace, amplitude=5pt}]
    (ewStartB) -- (ewEndB)
    node[black!50!green, midway, above=15pt, font=\scriptsize] {Endorsements window};

    % The 2k before end-of-epoch window.
    \coordinate (beeStart) at (\eend - 2*\K, -.5);
    \coordinate (beeEnd) at ($(\eend, -.5)$);
    \draw[blue, line width=4pt]
    (beeStart) -- (beeEnd);

    % Brace on above the 2k before end-of-epoch window.
    \coordinate (beeStartB) at ($(beeStart) - (0, 0.2)$);
    \coordinate (beeEndB) at ($(beeEnd) - (0, 0.2)$);
    \draw[thick, blue, decorate, decoration={brace, amplitude=5pt}]
    (beeEndB) -- (beeStartB)
    node[black!20!blue, midway, below=5pt, font=\scriptsize] {$2k$ slots before epoch ends};

    \draw[blue, line width=2pt] (\eend, 3pt) -- (\eend, -3pt);

    \draw[->] (\eend, -4pt) -- +(1, -1)
    node[right, blue, font=\scriptsize] {End of epoch};
  \end{tikzpicture}
  \caption{An update proposal confirmed too late}
  \label{fig:up-confirmed-too-late}
\end{figure}

\clearpage

\subsection{Deviations from the actual implementation}
\label{sec:deviation-actual-impl}

The current specification of the voting mechanism deviates from the actual
implementation, although it should be backwards compatible with the latter.
These deviations are required to simplify the voting and update mechanism
removing unnecessary features for a simplified setting, which will use the OBFT
consensus protocol with federated genesis key holders. This in turn, enables us
to remove any accidental complexity that might have been introduced in the
current implementation. The following subsections highlight the differences
between the this specification and the current implementation.

\subsubsection{Positive votes}
\label{sec:only-positive-votes}

Genesis keys can only vote (positively) for an update proposal. In the current
implementation stakeholders can vote for or against a proposal, which makes the
voting logic more complex:
\begin{itemize}
\item there are more cases to consider
\item the current voting validation rules allow voters to change their minds
  (by flipping their vote) at most once, which requires to keep track how a
  stake holder voted and how many times. Contrast this with
  Rule~\ref{eq:rule:voting} where we only need to keep track of the set of
  key-proposal-id's pairs.
\end{itemize}

\subsubsection{Alternative version numbers}
\label{sec:alt-version-numbers-constraints}

Alternative version numbers are only lexicographically constrained. The current
implementation seems to be dependent on the order in which the update proposals
arrive: given a new update proposal $\var{up}$, if a set $X$ of update
proposals with the same minor and major versions than $\var{up}$ exist, then
the alternative version of $\var{up}$ has to be one more than the maximum
alternative number of $X$. Not only this logic seems to be brittle since it
depends on the order of arrival of the update proposals, but it requires a more
complex check (which depends on state) to determine if a proposed version can
follow the current one. By being more lenient on the alternative versions of
update proposals we can simplify the version checking logic considerably.

\subsubsection{Cleanup}
\label{sec:up-cleanup}

Update proposals that are older than $u$ blocks w.r.t. the current block are
discarded from the state, along with their associated information. The current
implementation makes use of the implicit agreement rule and the epoch boundary
checks: this leads to plenty of different states for a proposal: active,
adopted, confirmed, competing, never-to-become-adopted, rejected, discarded. If
the cleanup of proposals can be done in the way specified here we will avoid a
great deal of cognitive complexity when reasoning about the update system.

\subsubsection{Adoption threshold}
\label{sec:adoption-threshold}

The current implementation adopts a proposal with version $\var{pv}$ if the
portion of block issuers' stakes, which issued blocks with this version, is
greater than the threshold given by:

\begin{lstlisting}
max spMinThd (spInitThd - (t - s) * spThdDecrement)
\end{lstlisting}

where:
\begin{itemize}
\item \lstinline{spMinThd} is a minimum threshold required for adoption.
\item \lstinline{spInitThd} is an initial threshold.
\item \lstinline{spThdDecrement} is the decrement constant of the initial
  threshold.
\end{itemize}

In this specification we only make use of a minimum adoption threshold,
represented by the protocol parameter $\var{upAdptThd}$ until it becomes clear
why a dynamic alternative is needed.

\subsubsection{No checks on unlock-stake-epoch parameter}
\label{sec:no-unlock-stake-epoch-check}

The rule of Figure~\ref{eq:rule:up-validity} does not check the
\lstinline{bvdUnlockStakeEpoch} parameter, since it will have a different
meaning in the handover phase: its use will be reserved for unlocking the
Ouroboros-BFT logic in the software.

\subsubsection{No grouping of proposals per-application name}
\label{sec:no-app-up-grouping}

It is unclear at this moment whether we need to model the applications names
and versions, so this aspect is not modeled in the present rules.

\subsubsection{No model of software-version}
\label{sec:no-model-software-version}

We do not model the
\href{https://github.com/input-output-hk/cardano-sl/blob/develop/docs/block-processing/us.md#software-version}{application
  name and the software version} of this application. It remains to be seen
whether this is part of the scope of this specification.


\subsubsection{Protocol-only vs software updates}
\label{sec:protocol-vs-software-updates}

We do not distinguish between protocol or software updates. The ledger only
cares about the mechanisms by which the protocol parameters are changed.

\subsubsection{Ignored attributes of proposals}

In Figure~\ref{fig:defs:update-proposals} the types
$\type{UpdData}$, and $\type{UpdAttr}$ are only needed to model the fact that
an update proposal must sign such data, however, we do not use them for any
other purpose in this formalization.

\subsection{Questions}
\label{sec:up-questions}

This temporary section contains unanswered questions about the formalization of
update mechanism:

\begin{itemize}
\item Do we need to model the $\var{scriptVersion}$?
\end{itemize}
